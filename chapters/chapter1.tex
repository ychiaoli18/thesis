\section{研究動機}
隨著科技的發展,我們越來越期望機器能夠理解問題並給予答案,進而解決一些生活上的事情。因此,我們需要一套問答系統 (Question Answering System) 來幫助我們快速的瀏覽資訊,並從中獲取答案。在過去有許多幫助我們回答問題,被開發出來並應用於產業中,如 Google Now 、Wolfram Alpha、Apple Siri等。
近年來隨著科技與網際網路的興起,維基百科 (Wikipedia) 等參考工具書正蓬勃地增加,人們追求更加直覺、便捷、有效率的資訊獲得途徑。在資料量大增的情況下,問答系統從大量的資料中直接擷取使用者想要的答案,僅只提供給使用者問題相關的答案,更加速了使用者獲得資訊的效率,因此如何讓機器能夠這些資料、理解、並回答問題,省去人們需要閱讀的時間,變成為重要的議題,而這即為問答系統。相較於其他的自然語言處理是提供給使用者文章、資源等,而問答系統有更困難的挑戰,如知識庫的廣大等,使得此問題更形艱辛。

近幾年更由於智慧型手機、穿戴式裝置的崛起與使用者需要在這些裝置上取得資訊的強烈需求,促使許多科技公司投入相關的研究,如 Facebook 有自己製造了如 bAbi 的人造資料集,希望機器在簡單的資料集上能有嬰兒一般學習能力、Google 整理了 CNN 以及每日郵報 (Daily Mail) 的新聞,以及 Microsoft 提供 MARCO 的資料集等,為的就是希望能在問答系統這塊能有顯著的突破。

問答系統是個複雜的自然語言處理問題,需要理解文字的意思。基本上,大多數(並非全部)的自然語言處理問題都能推廣成問答的問題,如機器翻譯 (Machine Translation),可以想成問一句中文話翻譯成英文是什麼、
%TODO
% 相較於資訊檢索系統,是提供給使用者文章、資源等,而問答系統


\section{研究方向}
本論文之目標在於閱讀文章的語意,並根據使用者的問句,試著去回答出答案,讓使用者更快速方便地獲得資訊,主要研究方向如下。
\begin{itemize}
    \item 傳統的自然語言處理需要對語言學有相關的知識,經過依賴關係解析 (Dependency Parsing) 和詞性標註 (Part of Speech Tagging, POS Tagging) 來分析句子的結構,並找出問句與文章之間的相關性。而深層類神經網路有好的推廣性,能省去其背後的分析,並且亦能有效的排除雜訊。
    \item 再者,為了能夠將字與字和句與句前後關係連結起來,採用位置編碼 (Position Encoding) 和時間遞歸神經網路 (Recurrent neural network) 中的閘門遞迴單元來產生一個句子的向量表示 (Vector Representation)。
    \item 由於有些句子可能是跟問題句子無關的,我們採用了專注式模型 (Attention Mechanism) 來找尋出重要的句子,忽視與問句無關的句子,藉此模擬人們的專注力。
    \item 藉由問句所通過編碼器的隱藏狀態 (hidden state) 當成記憶 (Memory),並靠著文章的向量表示來多次更新記憶。
    \item 最後以記憶當成解碼器的初始狀態,試著產生符合問句以及文章的句子。其中為了避免過度貼近 (Overfit) 測試資料,採用了權重遞減 (l2 regularization) 以及丟棄法 (Dropout)。
\end{itemize}

\section{章節安排}
本論文之章節安排如下:
\begin{itemize}
\itemsep -2pt
    \item 第二章:介紹本論文相關背景知識:包含深度類神經網路、時間遞歸神經網路、文字向量 (word2vec) 等等。
    \item 第三章:介紹如何以專注記憶式網路選擇適當的答案。
    \item 第四章:介紹如何以專注記憶式解碼器產生答案。
    \item 第五章:本論文之結論與未來研究方向。
\end{itemize}
