\section{研究動機}
隨著科技的發展,我們越來越期望機器能夠理解問題並給予答案,進而解決一些生活上的事情。因此,我們需要一套問答系統 (Question Answering System) 來幫助我們快速的瀏覽資訊,並從中獲取答案。在過去有許多幫助我們回答問題,被開發出來並應用於產業中,如 Google Now 、Wolfram Alpha、Apple Siri等。
近年來隨著科技與網際網路的興起,維基百科 (Wikipedia) 等參考工具書正蓬勃地增加,人們追求更加直覺、便捷、有效率的資訊獲得途徑。在資料量大增的情況下,問答系統從大量的資料中直接擷取使用者想要的答案,僅只提供給使用者問題相關的答案,更加速了使用者獲得資訊的效率,因此如何讓機器能夠這些資料、理解、並回答問題,省去人們需要閱讀的時間,變成為重要的議題,而這即為問答系統。相較於其他的自然語言處理,問答系統有更困難的挑戰,如知識庫的廣大等,使得此問題更形艱辛。
近年來更由於智慧型手機、穿戴式裝置的崛起與使用者需要在這些裝置上取得資訊的強烈需求,促使許多科技公司研究

相較於資訊檢索系統,是提供給使用者文章、資源等,而問答系統

\section{研究方向}
本論文之研究方向為使用

\section{相關研究}

\section{章節安排}
本論文之章節安排如下:

\begin{itemize}
\itemsep -2pt
    \item 第二章:
    \item 第三章:
    \item 第四章:
    \item 第五章:
\end{itemize}
